\documentclass[10pt,a4paper]{article}
\usepackage[T1]{fontenc}
\usepackage[utf8]{inputenc}
\usepackage{amssymb}
\usepackage{graphicx}
\usepackage[polish]{babel}
\usepackage{hyperref}
\usepackage[paper=a4paper,margin=1in]{geometry}
\usepackage{tabularx} 
\usepackage{helvet}
\usepackage{amsfonts}
\usepackage{amsthm}
\usepackage{mathtools}
\usepackage{algpseudocode}
\usepackage{tikz}
\usepackage{tkz-graph}
\usepackage{listings,xcolor}
\usepackage{scrextend}
\usepackage{float}
\usepackage{subcaption}
\renewcommand{\familydefault}{\sfdefault}
\setlength{\parindent}{0pt}
\newtheorem{definition}{Definicja}
\newtheorem{theorem}{Twierdzenie}
\newtheorem{lemma}{Lemat}
\newtheorem{invariant}{Niezmiennik}
\newtheorem{conculsion}{Wniosek}

\begin{document}
	\begin{titlepage}
		\newgeometry{top=1in,bottom=1in,right=1.5in,left=1.5in}
		\begin{center}
			{\fontsize{14}{12}\selectfont Politechnika Warszawska \\ Wydział Matematyki i Nauk Informacyjnych}
			
		\end{center}
		
		\vspace{1cm}
		\begin{center}
			\includegraphics[width=0.3\textwidth]{images/logo.png}
		\end{center}
		\vspace{3cm}
		
		\begin{center}
			\textbf{{\fontsize{26}{12}\selectfont Chromatyczna Teoria Grafów}}
			
			\vspace{2cm}
			\textbf{{\fontsize{22}{12}\selectfont Dokumentacja projektowa wstępna}}
			\vspace{1cm}
			
			\textbf{{\fontsize{13.5}{12}\selectfont Chimedshirchin Batjargal, Mateusz Rymuszka}}
			
			\vspace{6cm}
			\textbf{{\fontsize{13.5}{12}\selectfont \today}}
		\end{center}  
	\end{titlepage}
	
	%{\fontsize{13.5}{12}\selectfont
	%	\tableofcontents
	%	\vspace{1cm}
	%	{\renewcommand{\arraystretch}{2.0}
	%	
	%}}
	%
	%\newpage
	
	\section{Abstrakt}
	
	Przedmiotem projektu realizowanego w ramach przedmiotu Chromatyczna Teoria Grafów przez autorów tego dokumentu jest analiza problemu kolorowania warstwowego grafu. Zespół przygotuje, zaimplementuje oraz przetestuje działanie trzech algorytmy, które będą starały się pokolorować grafy wielowarstwowe w lepszy sposób niż naiwny algorytm duplikacji koloru na wiele warstw. W dokumentacji przedstawi teoretyczny opis problemu wraz z proponowanymi algorytmami, a następnie opisze sposób działania programu i przedstawi raport z testów na wybranych rodzajach grafów, podsumowując to wszystko wnioskami płynącymi z obserwacji.
	
	\section{Opis teoretyczny problemu}
	
	\begin{definition}\label{def:1}
		\textbf{Kolorowaniem $p$-warstwowym} grafu $G$ nazywamy takie przyporządkowanie $c: v \rightarrow 2^{C}$, gdzie każdemu wierzchołkowi $v \in V(G)$ przypisujemy podzbiór $C'$ zbioru kolorów $C$ taki, że $|C'| = p$.
	\end{definition}

	\begin{definition}\label{def:2}
		Kolorowanie $p$-warstwowe grafu $G$ nazywamy \textbf{właściwym (poprawnym, optymalnym)}, jeżeli dla dowolnego $v \in V(G)$ przecięcia zbioru kolorów tego wierzchołka i kolorów każdego jego sąsiada są zbiorami pustymi, tzn.
		\[ \forall u, v \in V(G) \quad \left\{u, v\right\} \in E(G) \implies c(u) \cap c(v) = \emptyset \]
	\end{definition}

	Ogólnie rzecz ujmując, kolorowanie wielowarstwowe grafu jest rozszerzeniem zwykłego kolorowania grafu na wiele wymiarów. Kolorowanie jednowarstwowe jest tożsame z klasyczną definicją kolorowania wierzchołkowego grafu.
	
	\begin{definition}
		\textbf{$p$-warstwową liczbą chromatyczną} grafu $G$ nazywamy najmniejsze $q$ takie, że istnieje poprawne $p$-warstwowe pokolorowanie grafu $G$ używające $q$ kolorów.\\
		\textbf{Chromatycznym współczynnikiem $p$-warstwowym} grafu $G$ nazywamy stosunek $p$-warstwowej liczby chromatycznej do liczby warstw.
	\end{definition}

	Okazuje się, że $\chi_{p}G \leq p \chi G$. Gdy weźmiemy bowiem dobre $\chi G$-pokolorowanie $c: V \rightarrow C$ grafu $G$ i dla każdego $v \in V$ przypiszemy mu $c'(v) = \left\{r \cdot |C| + c(v): r \in \left\{1,...,p\right\}\right\}$, to uzyskamy $p \chi G$-pokolorowanie $p$-warstwowe. Nazwijmy je pokolorowaniem $p$-warstwowym grafu $G$ \textbf{równoległym do klasycznego}.
	\\~\\
	Przykładem grafu, który dla którego zachodzi ostra nierówność, jest nieparzysty cykl o wielkości przekraczającej trzy wierzchołki. Takie kolorowanie będziemy nazywać \textbf{lepszym od klasycznego}.
	
	\begin{figure}[H]
		\centering
		\begin{tikzpicture}
			\node[circle, draw, fill=none] (1) [label=south west:{1,2}] at (0,0) {};
			\node[circle, draw, fill=none] (2) [label=south east:{3,4}] at (4,0) {};
			\node[circle, draw, fill=none] (3) [label=south east:{5,1}] at (5,4) {};
			\node[circle, draw, fill=none] (4) [label=north:{2,3}] at (2,6) {};
			\node[circle, draw, fill=none] (5) [label=south west:{4,5}] at (-1,4) {};
			\draw (1) -- (2) -- (3) -- (4) -- (5) -- (1);
		\end{tikzpicture}
		\caption{Przykład 2,5-pokolorowania 2-warstwowego dla grafu $C_{5}$}
	\end{figure}
	
	Problem polega na znalezieniu takiego $p$-warstwowego pokolorowania dla zadanego $p$, aby dla pewnych grafów (takich, gdzie jest to możliwe) $\chi_{p}G < p \chi G$.
	
	\section{Proponowane algorytmy}
	
	Ponieważ problem znalezienia $p$-warstwowego grafu jest NP-trudny (można to wykazać, rozwijając graf warstwowo, gdzie każdy wierzchołek tworzy $p$-krotną klikę, zachowując połączenia z wierzchołkami w każdej z warstw), nie możemy znaleźć takiego pokolorowania w czasie wielomianowym. Zespół skupi się na poszukiwaniu algorytmu aproksymacyjnego, który przy zastosowaniu odpowiedniego podejścia do problemu, a być może pewnych heurystyk, będzie w stanie osiągać rezultat dla niektórych grafów.
	\\~\\
	Wspólnym mianownikiem tych algorytmów powinna być troska, aby starać się kolorować "chaotycznie", tzn. nie powtarzać tych samych zbiorów kolorów w wierzchołkach niezależnych. Taki układ może nas potem zmusić do użycia większej liczby kolorów, a w efekcie uzyskamy pokolorowanie na $p \cdot \chi G$ kolorów.
	
	\subsection{Zmodyfikowany algorytm AMIS}
	
	Algorytm AMIS (\textit{ang. approximately maximum independent set}) jest jednym z efektywniejszych algorytmów wielomianowych kolorowania grafu. W klasycznym problemie, polega on na wyborze spośród niepokolorowanych wierzchołków zbioru niezależnego i pokolorowanie go na nowy kolor. \\~\\
	Algorytm można byłoby zaaplikować bez większych zmian (dopóty wierzchołek uznajemy za niepokolorowany, dopóki nie posiada zbioru $p$ kolorów przypisanych do niego). Istnieje jednak ryzyko, że wybierając te same zbiory niezależne w kolejnych iteracjach algorytmu znajdowania zachłannego zbiorów niezależnych, uzyskalibyśmy pokolorowanie równoległe do klasycznego. Zależy nam zatem, aby w kolejnych iteracjach zbiory niezależne jednocześnie  nie były tożsame oraz nie były rozłączne. \\~\\
	Zaproponowany algorytm będzie korzystał ze zoptymalizowanej wersji algorytmu GIS (\textit{ang. greedy independent sets}), w którym zamiast wyboru jednego wierzchołka, będziemy wybierać dla każdego wierzchołka, czy powinien się on znaleźć w nowym zbiorze wierzchołków niezależnych. Będziemy również pamiętać dla każdego wierzchołka, ile razy znalazł się on już w zbiorze niezależnym. Po wyborze wierzchołków do naszego zbioru niezależnego, będziemy analizować wszystkie krawędzie między kandydatami i wyrzucać jeden z wierzchołków. Będzie to wierzchołek, który znalazł się więcej lub mniej razy w zbiorze niezależnym (wyboru dokonujemy na zmianę).

	\subsection{Zmodyfikowany algorytm DSATUR}
	
	Algorytm DSATUR należy do rodziny algorytmów  W klasycznym problemie, polega on na wyborze spośród niepokolorowanych wierzchołków zbioru niezależnego i pokolorowanie go na nowy kolor. Zespół postanowił nieco zmodyfikować ten algorytm i zbadać jego działanie dla problemu kolorowania wielowarstwowego.
	\\~\\
	Saturację w klasycznej wersji tego algorytmu traktowaliśmy jako liczbę kolorów "zakazanych", tzn. ilość sąsiednich wierzchołków już pokolorowanych. Aby algorytm działał poprawnie dla wielu warstw, jednocześnie uwzględniając kolory znajdujące się u sąsiadów oraz w wierzchołku, należy zdefiniować nowy porządek. 
	\\~\\
	W celu analizy problemu, możemy rozpatrywać saturację wielorako; zdefiniujmy:
	\begin{itemize}
		\item saturację zewnętrzną, będącą ilością unikalnych kolorów użytych do pokolorowania wierzchołków sąsiednich, tzn.
		\[S^{OUT}_{G}(v) = \left|\bigcup_{u \in N_{G}(v)} c(u)\right|\]
		\item saturację wewnętrzną, będącą ilością kolorów użytych do pokolorowania wierzchołka, tzn.
		\[S^{IN}_{G}(v) = |c(v)|\]
		\item saturację całkowitą, będącą ilością unikalnych kolorów użytych do pokolorowania wierzchołków sąsiednich, tzn.
		\[S_{G}(v) = \left|\left(\bigcup_{u \in N_{G}(v)} c(u) \right) \cup c(v)\right|\]
	\end{itemize}
	Ponieważ $\forall u,v \in V(G) \quad {u, v} \in E(G) \implies c(u) \cap c(v) = \emptyset$ dla poprawnego $p$-warstwowego pokolorowania $c$ grafu $G$, mamy $S_{G}(v) = S^{OUT}_{G}(v) + S^{IN}_{G}(v)$. 
	\\~\\
	W każdym przypadku saturacją będzie saturacja całkowita i po niej będziemy w pierwszej kolejności sortować. W razie remisów wybieramy wierzchołek o większej saturacji wewnętrznej, gdyż później, gdy jego sąsiedzi mogą dostać kolejne kolory i będzie go ciężej pokolorować.

	\subsection{Trzeci algorytm}
	
	\section{Założenia techniczne}
	
	W wyborze języka, w którym zostanie stworzony program, zespół kierował się prostotą i przejrzystością implementacji, aby móc skupić się na teoretycznej części problemu. Ostatecznie, program zostanie napisany w języku Python - jest to bardzo prosty język, używany powszechnie do analizy złożonych algorytmów, posiadający duże wsparcie społeczności, bindingi do biblioteki graphviz (za pomocą której zespół chce zwizualizować wyniki działania algorytmów).
	\\~\\
	Projekt będzie obejmował:
	\begin{itemize}
		\item interfejs użytkownika (parser wejścia, komunikaty itp.)
		\item implementację wyżej wymienionych algorytmów 
		\item prezentację wyniku (wyjście tekstowe + wizualizacja za pomocą biblioteki graphviz)
		\item przykładowe benchmarki testujące skuteczność oraz czas wykonania na wybranych grafach
	\end{itemize}

	\vfill
	
	\begin{thebibliography}{0}
		\bibitem{kubale-article}
		Marek Kubale, \textit{Analiza efektywności algorytmów kolorowania grafów}, PTM 1980\\
		\url{https://wydawnictwa.ptm.org.pl/index.php/matematyka-stosowana/article/viewFile/1531/1457}\\
		\textit{(dostęp: \today)}
		
		\bibitem{kubale-book}
		Marek Kubale, \textit{Optymalizacja dyskretna. Modele i metody kolorownia grafów}, WNT 2002
	\end{thebibliography}
	

\end{document}